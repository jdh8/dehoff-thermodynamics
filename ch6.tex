\begin{@empty}
\section{Statistical thermodynamics}
\begin{problem}
\end{problem}

\begin{problem}
\end{problem}

\begin{problem}
\end{problem}

\begin{answer}
    \begin{enumerate}
        \item There are \( 4^2 = 16 \) microstates.
        \item The microstates are
            \( \left< \varepsilon_1, \varepsilon_1 \right> \),
            \( \left< \varepsilon_1, \varepsilon_2 \right> \),
            \( \left< \varepsilon_1, \varepsilon_3 \right> \),
            \( \left< \varepsilon_1, \varepsilon_4 \right> \),
            \( \left< \varepsilon_2, \varepsilon_1 \right> \),
            \( \left< \varepsilon_2, \varepsilon_2 \right> \),
            \( \left< \varepsilon_2, \varepsilon_3 \right> \),
            \( \left< \varepsilon_2, \varepsilon_4 \right> \),
            \( \left< \varepsilon_3, \varepsilon_1 \right> \),
            \( \left< \varepsilon_3, \varepsilon_2 \right> \),
            \( \left< \varepsilon_3, \varepsilon_3 \right> \),
            \( \left< \varepsilon_3, \varepsilon_4 \right> \),
            \( \left< \varepsilon_4, \varepsilon_1 \right> \),
            \( \left< \varepsilon_4, \varepsilon_2 \right> \),
            \( \left< \varepsilon_4, \varepsilon_3 \right> \),
            and \( \left< \varepsilon_4, \varepsilon_4 \right> \).
        \item The macrostates are
            \( \left\{ \varepsilon_1, \varepsilon_1 \right\} \),
            \( \left\{ \varepsilon_1, \varepsilon_2 \right\} \),
            \( \left\{ \varepsilon_1, \varepsilon_3 \right\} \),
            \( \left\{ \varepsilon_1, \varepsilon_4 \right\} \),
            \( \left\{ \varepsilon_2, \varepsilon_2 \right\} \),
            \( \left\{ \varepsilon_2, \varepsilon_3 \right\} \),
            \( \left\{ \varepsilon_2, \varepsilon_4 \right\} \),
            \( \left\{ \varepsilon_3, \varepsilon_3 \right\} \),
            \( \left\{ \varepsilon_3, \varepsilon_4 \right\} \),
            and \( \left\{ \varepsilon_4, \varepsilon_4 \right\} \).
    \end{enumerate}
\end{answer}

\begin{problem}
\end{problem}

\begin{problem}
\end{problem}

\begin{answer}
    \begin{enumerate}
        \item
            \[ 0 + 0 + 1 + 2 + 4 + 2 + 1 + 0 + 0 + 0 = 10 \]
            \[ 0 + 1 + 1 + 2 + 2 + 2 + 1 + 1 + 0 + 0 = 10 \]
            These states has the same energy.
        \item
            \[ \Omega_1 = \frac{10!}{5!2!2!1!} \]
            \[ \Omega_2 = \frac{10!}{3!4!3!} \]
            \[ \frac{\Omega_2}{\Omega_1} = \frac{5!2!2!1!}{3!4!3!} = \frac95 \]
            State II has higher entropy.
        \item State II is more likely to be observed.
    \end{enumerate}
\end{answer}

\begin{problem}
\end{problem}

\begin{problem}
\end{problem}

\begin{answer}
    The initial state is
    \[ \left\{ 14, 18, 27, 38, 51, 78, 67, 54, 32, 27, 23, 20, 19, 17, 15 \right\}.\]
    The final state is
    \[ \left\{ 14, 18, 26, 37, 49, 78, 68, 55, 34, 29, 24, 20, 18, 16, 14 \right\}.\]

    Sort the elements to make duplicate particles stand out.  The initial state
    is
    \[ \left\{ 14, 15, 17, 18, 19, 20, 23, 27, 27, 32, 38, 51, 54, 67, 78 \right\} \]
    \[ \Omega_0 = \frac{10!}{2!}.\]

    The final state is
    \[ \left\{ 14, 14, 16, 18, 18, 20, 24, 26, 29, 34, 37, 49, 55, 68, 78 \right\} \]
    \[ \Omega_1 = \frac{10!}{2!2!}.\]

    Therefore, the difference in entropy is
    \begin{align*}
        \Delta S &= R \ln \frac{\Omega_1}{\Omega_0} = R \ln\frac12 \\
        &= \SI{-5.763}{\joule/\mol.\kelvin} \\
        &= \SI{-9.572e-24}{\joule/\kelvin}.
    \end{align*}
\end{answer}
\end{@empty}
