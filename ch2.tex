\begin{@empty}
\section{The structure of thermodynamics}
\begin{problem}
    Classify the following thermodynamic systems in the five categories defined
    in Section 2.1:
    \begin{enumerate}
        \item A solid bar of copper.
        \item A glass of ice water.
        \item A yttrium stabilized zirconia furnace.
        \item A Styrofoam coffee cup.
        \item A eutectic alloy turbine blade rotating at 20,000 r/min.
    \end{enumerate}

    You may find it necessary to qualify your answer by defining the system
    more precisely; state your assumptions.
\end{problem}

\begin{remark}
    \renewcommand{\theenumi}{\arabic{enumi}}
    The aforementioned five categories are:
    \begin{enumerate}
        \item Unary vs.\ multicomponent
        \item Homogeneous vs.\ heterogeneous
        \item Closed vs.\ open
        \item Nonreacting vs.\ reacting
        \item Otherwise simple vs.\ complex
    \end{enumerate}
\end{remark}

\begin{answer}
    TODO
\end{answer}

\begin{problem}
\end{problem}

\begin{problem}
    Determine which of the following properties of a thermodynamic system are
    extensive properties and which are intensive:
    \begin{enumerate}
        \item The mass density.
        \item The molar density.
        \item The number of gram atoms of aluminum in a chunk of alumina.
        \item The potential energy of a system in a gravitational field.
        \item The molar concentration of NaCl in a salt solution.
        \item The heat absorbed by the gas in a cylinder when it is compressed.
    \end{enumerate}
\end{problem}

\begin{answer}
    (Note that \textbf{densities} convert extensive properties into
    intensive ones.)
    \begin{enumerate}
        \item Intensive
        \item Intensive
        \item Extensive
        \item Extensive
        \item Intensive
        \item Extensive
    \end{enumerate}
\end{answer}

\begin{problem}
    Why is heat a process variable?
\end{problem}

\begin{problem}
    Write the total differential of the function:
    \[ z = 12 u^3 v \cos(x) \]

    \begin{enumerate}
        \item Identify the coefficients of the three differentials in this
            expression as appropriate partial derivatives.
        \item Show that three Maxwell relations hold among these coefficients.
    \end{enumerate}
\end{problem}

\begin{answer}
    \begin{enumerate}
        \item Total derivative can be constructed from partial derivatives.
            \begin{align*}
                \frac{\partial z}{\partial u} &= 36 u^2 v \cos(x) \\
                \frac{\partial z}{\partial v} &= 12 u^3 \cos(x) \\
                \frac{\partial z}{\partial x} &= -12 u^3 v \sin(x)
            \end{align*}

            \[ dz = 36 u^2 v \cos(x) \,du
                + 12 u^3 \cos(x) \,dv
                - 12 u^3 v \sin(x) \,dx.\]

        \item Maxwell relations are direct corollaries of Schwarz's theorem.
            \[
                \begin{gathered}
                    \frac\partial{\partial v}\frac{\partial z}{\partial u}
                        = 36 u^2 \cos(x)
                        = \frac\partial{\partial u}\frac{\partial z}{\partial v} \\
                    \frac\partial{\partial x}\frac{\partial z}{\partial u}
                        = -36 u^2 v \sin(x)
                        = \frac\partial{\partial u}\frac{\partial z}{\partial x} \\
                    \frac\partial{\partial x}\frac{\partial z}{\partial v}
                        = -12 u^3 \sin(x)
                        = \frac\partial{\partial v}\frac{\partial z}{\partial x} \\
                \end{gathered}
            \]
    \end{enumerate}
\end{answer}

\begin{problem}
\end{problem}
\end{@empty}
