\documentclass[a4paper, 12pt]{article}
\usepackage{amsmath, amsthm}
\usepackage{geometry}

\newcommand{\Left}{\mathopen{}\mathclose\bgroup\left}
\newcommand{\Right}{\aftergroup\egroup\right}

\theoremstyle{definition}
\newtheorem{problem}{Problem}[section]

\theoremstyle{remark}
\newtheorem*{remark}{Remark}
\newtheorem*{answer}{Answer}

\renewcommand{\theenumi}{\alph{enumi}}

\title{DeHoff's Thermodynamics}
\author{Chen-Pang He}

\begin{document}
\maketitle
This document contains unofficial answers to homework problems in Robert
DeHoff's \textit{Thermodynamics in Materials Science}, 2nd ed.

\section{Why study thermodynamics?}
No homework problems are in this chapter.

\section{The structure of thermodynamics}
\begin{problem}
    Classify the following thermodynamic systems in the five categories defined
    in Section 2.1:
    \begin{enumerate}
        \item A solid bar of copper.
        \item A glass of ice water.
        \item A yttrium stabilized zirconia furnace.
        \item A Styrofoam coffee cup.
        \item A eutectic alloy turbine blade rotating at 20,000 r/min.
    \end{enumerate}

    You may find it necessary to qualify your answer by defining the system
    more precisely; state your assumptions.

    \begin{remark}
        \renewcommand{\theenumi}{\arabic{enumi}}
        The aforementioned five categories are:
        \begin{enumerate}
            \item Unary vs.\ multicomponent
            \item Homogeneous vs.\ heterogeneous
            \item Closed vs.\ open
            \item Nonreacting vs.\ reacting
            \item Otherwise simple vs.\ complex
        \end{enumerate}
    \end{remark}

    \begin{answer}
        TODO
    \end{answer}
\end{problem}

\begin{problem}
\end{problem}

\begin{problem}
    Determine which of the following properties of a thermodynamic system are
    extensive properties and which are intensive:
    \begin{enumerate}
        \item The mass density.
        \item The molar density.
        \item The number of gram atoms of aluminum in a chunk of alumina.
        \item The potential energy of a system in a gravitational field.
        \item The molar concentration of NaCl in a salt solution.
        \item The heat absorbed by the gas in a cylinder when it is compressed.
    \end{enumerate}

    \begin{answer}
        (Note that \textbf{densities} convert extensive properties into
        intensive ones.)
        \begin{enumerate}
            \item Intensive
            \item Intensive
            \item Extensive
            \item Extensive
            \item Intensive
            \item Extensive
        \end{enumerate}
    \end{answer}
\end{problem}

\begin{problem}
    Why is heat a process variable?
\end{problem}

\begin{problem}
    Write the total differential of the function:
    \[ z = 12 u^3 v \cos(x) \]

    \begin{enumerate}
        \item Identify the coefficients of the three differentials in this
            expression as appropriate partial derivatives.
        \item Show that three Maxwell relations hold among these coefficients.
    \end{enumerate}
    
    \begin{remark}
        \[ \frac{d}{dx} x^n = n x^{n - 1} \]
        \[ \frac{d}{dx} \cos(x) = -\sin(x).\]
    \end{remark}

    \begin{answer}
        \begin{align*}
            \frac{\partial z}{\partial u} &= 36 u^2 v \cos(x) \\
            \frac{\partial z}{\partial v} &= 12 u^3 \cos(x) \\
            \frac{\partial z}{\partial x} &= -12 u^3 v \sin(x) \\
        \end{align*}

        \[ dz = 36 u^2 v \cos(x) \,du
            + 12 u^3 \cos(x) \,dv
            - 12 u^3 v \sin(x) \,dx.\]

        \[
            \begin{gathered}
                \frac\partial{\partial v}\frac{\partial z}{\partial u}
                    = 36 u^2 \cos(x)
                    = \frac\partial{\partial u}\frac{\partial z}{\partial v} \\
                \frac\partial{\partial x}\frac{\partial z}{\partial u}
                    = -36 u^2 v \sin(x)
                    = \frac\partial{\partial u}\frac{\partial z}{\partial x} \\
                \frac\partial{\partial x}\frac{\partial z}{\partial v}
                    = -12 u^3 \sin(x)
                    = \frac\partial{\partial v}\frac{\partial z}{\partial x} \\
            \end{gathered}
        \]
    \end{answer}
\end{problem}

\begin{problem}
\end{problem}

\section{The laws of thermodynamics}

\section{Thermodynamic variables and relations}
\begin{problem}
    Write out the combined statements of the first and second laws for the
    energy functions, $U = U(S, V)$, $H = H(S, P)$, $F = F(T, V)$ and $G = G(T,
    P)$.  Assume $\delta W'$ is zero:
    \begin{enumerate}
        \item Write out all eight coefficient relations.
        \item Derive all four Maxwell relations.
    \end{enumerate}
    for these equations.

    \begin{remark}
        When deriving Maxwell relations, use Schwarz's theorem:
        \[ \frac\partial{\partial y} \frac{\partial z}{\partial x}
            = \frac\partial{\partial x} \frac{\partial z}{\partial y}
            = \frac{\partial^2 z}{\partial x \partial y}.\]
    \end{remark}

    \begin{answer}
        \begin{enumerate}
            \item According to definitions,
                \begin{align*}
                    H &= U + PV \\
                    F &= U - TS \\
                    G &= H - TS \\
                    G &= U + PV - TS. \tag{?}
                \end{align*}

                By the first and second laws,
                \begin{align*}
                    dU &= TdS - PdV \\
                    dH &= TdS + VdP \\
                    dF &= -SdT - PdV \\
                    dG &= -SdT + VdP.
                \end{align*}

            \item
                \[
                    \frac{\partial^2 U}{\partial S \partial V}
                    = \left( \frac{\partial T}{\partial V} \right)_S
                    = -\left( \frac{\partial P}{\partial S} \right)_V
                \]
        \end{enumerate}
    \end{answer}
\end{problem}
\end{document}
